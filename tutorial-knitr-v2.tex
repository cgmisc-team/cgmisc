\documentclass[12pt,a4paper,oneside]{article}
\usepackage[usenames,dvipsnames]{xcolor}
\usepackage{graphicx}
\usepackage{amsmath}
\usepackage{booktabs}



\begin{document}





\title{\color{TealBlue} cgmisc \\\ \normalsize Package containing various functions useful in computational genetics, especially in genome-wide association studies}
\author{\color{Orange}Marcin Kierczak}
\maketitle

\newpage

\section*{Introduction}
\subsection*{Synopsis}
\noindent Package cgmisc contains miscellaneous functions, hopefully useful for extending genome-wide association study (GWAS) analyses.
\subsection*{Getting help}
\noindent Like every other R function, the functions provided in this package are documented in the standard R-help (Rd) format and can be easily accessed by issuing \textbf{help}() or its shorter version, \textbf{?} function. For instance, if you want to get more information on how to use the clump.markers() function, type either help(clumpmarkers) or ?clump.markers and press return/enter. To see this document from within R you type vignette(`cgmisc`).
\subsection*{Purpose of this document}
\noindent This document aims at presenting how to use functions provided in this package in a typical GWAS data analyses workflow. It is, however, not pretending to be a GWAS tutorial as such.
\subsection*{Conventions}
\begin{itemize}
\item{All R commands are written in terminal type: myfun(foo=T, bar=54)}
\item{In the above example: \textit{myfun} is a function and both \textit{foo} and \textit{bar} are its arguments}
\end{itemize}

\section*{Working with \texttt{cgmisc}}
\subsection*{Installation}
\noindent In order to install \texttt{cgmisc}, you either use one of the R GUIs (native R GUI, RStudio etc.) or type the following command:

\begin{knitrout}\footnotesize
\definecolor{shadecolor}{rgb}{0.969, 0.969, 0.969}\color{fgcolor}\begin{kframe}
\begin{alltt}
 \hlkwd{install.packages}\hlstd{(}\hlstr{"cgmisc"}\hlstd{,} \hlkwc{repos}\hlstd{=}\hlstr{""}\hlstd{)}
\end{alltt}
\end{kframe}
\end{knitrout}

\noindent Functions in the \texttt{cgmisc} package often complement or use \texttt{GenABEL} package functions and data structures.\texttt{GenABEL} is an excellent and widely-used R package for performing genome-wide association studies and much more... Therefore \texttt{GenABEL} will be loaded automagically when loading cgmisc. You can load \texttt{cgmisc} package by as follows:

\begin{knitrout}\footnotesize
\definecolor{shadecolor}{rgb}{0.969, 0.969, 0.969}\color{fgcolor}\begin{kframe}
\begin{alltt}
\hlkwd{require}\hlstd{(}\hlstr{"cgmisc"}\hlstd{)}
\end{alltt}


{\ttfamily\noindent\itshape\color{messagecolor}{\#\# Loading required package: cgmisc\\\#\# \\\#\#\ \ Package cgmisc contains miscellaneous functions, useful for extending\\\#\# genome-wide association study (GWAS) analyses. \\\#\# \\\#\# Package Name: cgmisc \\\#\#\ \ Version: 2.9.1 \\\#\#\ \ Date: 2014-07-29 \\\#\#\ \ Author: Marcin Kierczak <marcin.kierczak@imbim.uu.se> \\\#\#\ \ License GPL (>=2.10) \\\#\# \\\#\#\ \ Package contains various functions useful in computational\\\#\#\ \ \ \  genetics, especially in genome-wide association studies.}}\end{kframe}
\end{knitrout}

\noindent After having loaded the package it is time to load some data:

\begin{knitrout}\footnotesize
\definecolor{shadecolor}{rgb}{0.969, 0.969, 0.969}\color{fgcolor}\begin{kframe}
\begin{alltt}
  \hlkwd{load}\hlstd{(}\hlstr{'data/data.rda'}\hlstd{)}
\end{alltt}
\end{kframe}
\end{knitrout}

\subsection*{Association Analysis}
\noindent Some of \texttt{cgmisc} functions use data which are the result of GWAS analyses. Let's perform GWAS on our data to obtain \texttt{GenABEL} \texttt{scan.gwaa-class} object : 

\begin{knitrout}\footnotesize
\definecolor{shadecolor}{rgb}{0.969, 0.969, 0.969}\color{fgcolor}\begin{kframe}
\begin{alltt}
\hlstd{an0} \hlkwb{<-} \hlkwd{qtscore}\hlstd{(response} \hlopt{~} \hlstd{sex,} \hlkwc{data} \hlstd{= data)}
\end{alltt}


{\ttfamily\noindent\color{warningcolor}{\#\# Warning: 1 observations deleted due to missingness}}\end{kframe}
\end{knitrout}

\noindent And have a look at top 5 markers

\begin{knitrout}\footnotesize
\definecolor{shadecolor}{rgb}{0.969, 0.969, 0.969}\color{fgcolor}\begin{kframe}
\begin{alltt}
\hlkwd{summary}\hlstd{(an0,} \hlkwc{top} \hlstd{=} \hlnum{5}\hlstd{)}
\end{alltt}
\begin{verbatim}
## Summary for top 5 results, sorted by P1df
##                 Chromosome Position Strand A1 A2   N    effB se_effB
## BICF2P1063345           34 40399702      u  T  G 206 -0.2834 0.06954
## BICF2P682714             1 15399848      u  A  G 189  0.5887 0.14714
## BICF2G630569243          6 80458945      u  C  A 206  0.3071 0.07797
## BICF2S2366791            6 70667322      u  C  T 206  0.2628 0.06682
## BICF2G630450144         34 40416964      u  A  G 206 -0.2705 0.06933
##                 chi2.1df      P1df   effAB   effBB chi2.2df      P2df
## BICF2P1063345      16.60 4.609e-05 -0.3644 -0.5354    17.51 1.579e-04
## BICF2P682714       16.01 6.309e-05  0.5912  1.1515    16.01 3.339e-04
## BICF2G630569243    15.51 8.216e-05  0.2904  0.6358    15.57 4.167e-04
## BICF2S2366791      15.46 8.410e-05  0.4558  0.4732    20.15 4.202e-05
## BICF2G630450144    15.22 9.573e-05 -0.3339 -0.5161    15.77 3.756e-04
##                     Pc1df
## BICF2P1063345   7.357e-05
## BICF2P682714    9.911e-05
## BICF2G630569243 1.274e-04
## BICF2S2366791   1.302e-04
## BICF2G630450144 1.472e-04
\end{verbatim}
\end{kframe}
\end{knitrout}


\noindent Once this is done, we can proceed with \texttt{cgmisc} functions.

\newpage

\section*{Functions}
\subsection*{Plot.Manhattan.LD}
\noindent The \texttt{plot.Manhattan.LD} function allows you to visualize the LD pattern in a genome fragment on an enchanced Manhattan plot. You select one marker, typically the one with the strongest association to the analysed trait and all other markers in the region are coloured according to the degree of linkage disequilibrium with this index marker. 

































